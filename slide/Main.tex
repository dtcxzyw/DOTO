\documentclass[aspectratio=43,compress]{beamer}
\usetheme{material}

\usepackage{fontspec}
\usepackage{xunicode,xltxtra}
\usepackage[AutoFakeBold=true,AutoFakeSlant=true,CheckSingle=true,PlainEquation=true,PunctStyle=kaiming]{xeCJK}
\defaultfontfeatures{Mapping=tex-text}
\XeTeXlinebreaklocale "zh"
\XeTeXlinebreakskip = 0pt plus 1pt minus 0.1pt
\setCJKmainfont[BoldFont=SourceHanSansCN-Bold.otf,
ItalicFont=SourceHanSansCN-Light.otf]
{SourceHanSansCN-Regular.otf}
\setCJKmonofont{Source Han Sans CN}
\setCJKsansfont{Source Han Sans CN}
\usepackage[english]{babel}
\usepackage[T1]{fontenc}
\usepackage{CJKfntef}

\title{AI实现分享}
\date{May 12, 2019}
\author[dtcxzyw]{dtcxzyw}

\useLightTheme
\usePrimaryBlue
\useAccentIndigo

\begin{document}

\begin{frame}
\titlepage
\end{frame}

\begin{frame}{目录}
\begin{card}
\tableofcontents
\end{card}
\end{frame}

\section{安全措施}
\begin{frame}{安全措施}
\begin{card}
我的个性签名是——安全第一。

由于我不会附加gdb到AI上调试,只能通过SDK的debug接口输出调试信息。

下面从三个角度防范bug:
\begin{itemize}
    \item 执行时:拦截并报告无效操作,尽可能记录调用位置。
    \item 执行后:在下一次决策时检查上一次决策是否有效,统计性能信息
    (掉帧过多,技能利用率过低都可能是bug引起的)
    \item 无药可救时:吞下在计算过程中出现的异常,让AI尽可能在下一帧继续工作。
\end{itemize}

这些无关AI决策的代码帮助我快速发现并修复bug。
\end{card}
\end{frame}

\begin{frame}
\begin{card}[安全措施——执行时]
将所有的执行指令集中到SafeOp名称空间中进行,它将检查每个参数的有效性。

至于调用位置,C++20中有std::source\_location,调用时传进去。

注意它并不能代替决策时对参数有效性的检查,要把它独立于AI逻辑之外。
\end{card}
\begin{card}[安全措施——执行后]
存储上一帧的决策与调用方,检查该帧的状态是否为期望状态。

此外还统计掉帧,检查行走是否被卡死,统计移动、火球、陨石阵的利用率。
\end{card}
\end{frame}

\begin{frame}{安全措施——无药可救时}
\begin{card}[C++ Exception]
以帧为处理粒度,在playerAI函数套一个就完事了。
\end{card}
\begin{card}[signal]
段错误会触发SIGSEGV信号,可以替换信号处理函数。
\end{card}
\begin{card}[errno]
对于数学函数的异常,只能在单帧处理结束后检查errno。
\end{card}
\end{frame}

\section{AI决策相关}
\subsection{局面分析}
\begin{frame}{局面分析}
\begin{card}
分析全局局面对于决策是很有必要的。

Analyzer提供了如下信息:
\begin{itemize}
\item 上一次bonus消失的时间
\item 上一帧敌人移动的方向
\item 检查单位向目标的行进速度是否过慢(用于突击)
\item 列出最近攻击该单位的敌人编号
\end{itemize}
\end{card}
\end{frame}

\subsection{采样系统}
\begin{frame}{采样系统}
\begin{card}[起源]
    当没有数学模型来指导决策怎么办?随机!!!

    方法简单粗暴:随机满足要求的决策,选取估值最高的。
\end{card}
\begin{card}[进阶]
    但是一堆if和for的嵌套好难写!!!还有更复杂的需求:
    \begin{itemize}
        \item 以一定概率忽视某一考虑项
        \item 各类估值按照字典序比较
        \item 尽可能选取满足某一条件的元素
    \end{itemize}
\end{card}
\end{frame}

\begin{frame}{GROES}
\begin{card}
将这些需求归类,我搞出了一个通用的采样系统!!!

采样系统由3类估值函数组成:
\begin{itemize}
    \item request:强制要求,无视优先级
    \item optional:可选要求,会被转换为estimate
    \item estimate:估值,将Point映射为double,按照字典序比较
\end{itemize}
\end{card}
\end{frame}

\begin{frame}{GROES实现}
\begin{card}
感谢伟大的lambda、std::function、Templates和Proxy Classes!!!
\CJKsout{但这并不能改变我将来转Rust的决定。}

整个系统由PointSampler类维护,使用Generator来构造PointSampler,
然后添加一堆的ROE描述需求,最后调用sample方法得到结果。

像Parallel Patterns Library的一连串.then(),可以使用链式调用一句搞定sample。

对于那些可有可无的需求,增加一个apply方法让lambda修改自己以避免打断链式调用。

对于bonus这种需要精确降落的移动,可以通过hotPoint来引导选择。
\end{card}
\end{frame}

\begin{frame}{GROES扩展}
\begin{cardTiny}
选择攻击/跟踪对象时,还有从一些id中选出一个或几个id的需求。

再写一个DiscreteSampler来满足这个需求。
\end{cardTiny}
\begin{cardTiny}
estimate的优先级不应过高,除非使用一些舍入/分段函数嵌套在外面。

Sampler自带一些Helper~Function和重载运算符避免写lambda。

Map和Attack模块提供了基本的Sampler以确保决策有效。

它们的Generator使用的不是std::mt19937\_64而是Halton序列。
序列生成代码是从我的CUDA软件渲染器上扒下来的。
\end{cardTiny}
\end{frame}

\begin{frame}{GROES表达}
\begin{cardTiny}
是不是很像Scratch???
\end{cardTiny}
\centering
\cardImg{example.png}{300.0pt}
\end{frame}
\subsection{任务分配}
\begin{frame}{任务分配}
\begin{card}
经典的1-3-1分配,在输送单位安全时,空闲的单位会返回去输送新的水晶。

bonus处的单位是永久职业的,不允许残血的单位到那里去送死。
\end{card}
\end{frame}
\subsection{移动}
\begin{frame}{移动}
\begin{card}
\begin{itemize}
\item A*仅提供前进方向指导。
\item 优先躲避meteor,其次是fireball。
\item 有时会与己方单位/敌方单位保持距离。
\item bonus处使用保守策略。
\end{itemize}
\end{card}
\end{frame}
\subsection{攻击}
\begin{frame}{攻击}
\begin{card}
\begin{itemize}
\item 以消灭敌人有生力量为目标。
\item 尽可能同时影响到多个敌人。
\item 选取近的敌人作为目标。
\item meteor仅用于阻挡敌人前进。
\item 在无法打到目标时,识别转角并封锁。
\item bonus处使用保守策略。
\end{itemize}
\end{card}
\end{frame}
\subsection{未使用的idea}
\begin{frame}{未使用的idea}
\begin{card}
\begin{itemize}
    \item 远程扫射(看不出效果)
    \item 预测攻击估值函数(准确率30\%以下)
    \item 惰性分块Dijkstra(已被A*代替)
    \item 寻找掩体(不会写)
    \item 平行同步齐射(射速太慢)
    \item 惯性预瞄(虽然惯性预测的准确率在60\%到80\%左右,但是剩下的miss
    应该都是敌人躲火球的时候发生的)
\end{itemize}
\end{card}
\end{frame}

\section{End}
\begin{frame}{End}
\begin{card}
感谢大家的聆听!

感谢秦岳同学、吕紫剑同学在APIO、THUWC、THUSC上的宣传。

该slides的模板:

{\footnotesize \url{https://github.com/edasubert/beamerMaterialDesign}}
\end{card}
\end{frame}
\end{document}
